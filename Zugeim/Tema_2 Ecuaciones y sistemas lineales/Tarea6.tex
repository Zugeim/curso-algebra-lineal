% Options for packages loaded elsewhere
\PassOptionsToPackage{unicode}{hyperref}
\PassOptionsToPackage{hyphens}{url}
%
\documentclass[
]{article}
\usepackage{lmodern}
\usepackage{amssymb,amsmath}
\usepackage{ifxetex,ifluatex}
\ifnum 0\ifxetex 1\fi\ifluatex 1\fi=0 % if pdftex
  \usepackage[T1]{fontenc}
  \usepackage[utf8]{inputenc}
  \usepackage{textcomp} % provide euro and other symbols
\else % if luatex or xetex
  \usepackage{unicode-math}
  \defaultfontfeatures{Scale=MatchLowercase}
  \defaultfontfeatures[\rmfamily]{Ligatures=TeX,Scale=1}
\fi
% Use upquote if available, for straight quotes in verbatim environments
\IfFileExists{upquote.sty}{\usepackage{upquote}}{}
\IfFileExists{microtype.sty}{% use microtype if available
  \usepackage[]{microtype}
  \UseMicrotypeSet[protrusion]{basicmath} % disable protrusion for tt fonts
}{}
\makeatletter
\@ifundefined{KOMAClassName}{% if non-KOMA class
  \IfFileExists{parskip.sty}{%
    \usepackage{parskip}
  }{% else
    \setlength{\parindent}{0pt}
    \setlength{\parskip}{6pt plus 2pt minus 1pt}}
}{% if KOMA class
  \KOMAoptions{parskip=half}}
\makeatother
\usepackage{xcolor}
\IfFileExists{xurl.sty}{\usepackage{xurl}}{} % add URL line breaks if available
\IfFileExists{bookmark.sty}{\usepackage{bookmark}}{\usepackage{hyperref}}
\hypersetup{
  pdftitle={Tarea 6},
  pdfauthor={Imanol},
  hidelinks,
  pdfcreator={LaTeX via pandoc}}
\urlstyle{same} % disable monospaced font for URLs
\usepackage[margin=1in]{geometry}
\usepackage{color}
\usepackage{fancyvrb}
\newcommand{\VerbBar}{|}
\newcommand{\VERB}{\Verb[commandchars=\\\{\}]}
\DefineVerbatimEnvironment{Highlighting}{Verbatim}{commandchars=\\\{\}}
% Add ',fontsize=\small' for more characters per line
\usepackage{framed}
\definecolor{shadecolor}{RGB}{248,248,248}
\newenvironment{Shaded}{\begin{snugshade}}{\end{snugshade}}
\newcommand{\AlertTok}[1]{\textcolor[rgb]{0.94,0.16,0.16}{#1}}
\newcommand{\AnnotationTok}[1]{\textcolor[rgb]{0.56,0.35,0.01}{\textbf{\textit{#1}}}}
\newcommand{\AttributeTok}[1]{\textcolor[rgb]{0.77,0.63,0.00}{#1}}
\newcommand{\BaseNTok}[1]{\textcolor[rgb]{0.00,0.00,0.81}{#1}}
\newcommand{\BuiltInTok}[1]{#1}
\newcommand{\CharTok}[1]{\textcolor[rgb]{0.31,0.60,0.02}{#1}}
\newcommand{\CommentTok}[1]{\textcolor[rgb]{0.56,0.35,0.01}{\textit{#1}}}
\newcommand{\CommentVarTok}[1]{\textcolor[rgb]{0.56,0.35,0.01}{\textbf{\textit{#1}}}}
\newcommand{\ConstantTok}[1]{\textcolor[rgb]{0.00,0.00,0.00}{#1}}
\newcommand{\ControlFlowTok}[1]{\textcolor[rgb]{0.13,0.29,0.53}{\textbf{#1}}}
\newcommand{\DataTypeTok}[1]{\textcolor[rgb]{0.13,0.29,0.53}{#1}}
\newcommand{\DecValTok}[1]{\textcolor[rgb]{0.00,0.00,0.81}{#1}}
\newcommand{\DocumentationTok}[1]{\textcolor[rgb]{0.56,0.35,0.01}{\textbf{\textit{#1}}}}
\newcommand{\ErrorTok}[1]{\textcolor[rgb]{0.64,0.00,0.00}{\textbf{#1}}}
\newcommand{\ExtensionTok}[1]{#1}
\newcommand{\FloatTok}[1]{\textcolor[rgb]{0.00,0.00,0.81}{#1}}
\newcommand{\FunctionTok}[1]{\textcolor[rgb]{0.00,0.00,0.00}{#1}}
\newcommand{\ImportTok}[1]{#1}
\newcommand{\InformationTok}[1]{\textcolor[rgb]{0.56,0.35,0.01}{\textbf{\textit{#1}}}}
\newcommand{\KeywordTok}[1]{\textcolor[rgb]{0.13,0.29,0.53}{\textbf{#1}}}
\newcommand{\NormalTok}[1]{#1}
\newcommand{\OperatorTok}[1]{\textcolor[rgb]{0.81,0.36,0.00}{\textbf{#1}}}
\newcommand{\OtherTok}[1]{\textcolor[rgb]{0.56,0.35,0.01}{#1}}
\newcommand{\PreprocessorTok}[1]{\textcolor[rgb]{0.56,0.35,0.01}{\textit{#1}}}
\newcommand{\RegionMarkerTok}[1]{#1}
\newcommand{\SpecialCharTok}[1]{\textcolor[rgb]{0.00,0.00,0.00}{#1}}
\newcommand{\SpecialStringTok}[1]{\textcolor[rgb]{0.31,0.60,0.02}{#1}}
\newcommand{\StringTok}[1]{\textcolor[rgb]{0.31,0.60,0.02}{#1}}
\newcommand{\VariableTok}[1]{\textcolor[rgb]{0.00,0.00,0.00}{#1}}
\newcommand{\VerbatimStringTok}[1]{\textcolor[rgb]{0.31,0.60,0.02}{#1}}
\newcommand{\WarningTok}[1]{\textcolor[rgb]{0.56,0.35,0.01}{\textbf{\textit{#1}}}}
\usepackage{graphicx}
\makeatletter
\def\maxwidth{\ifdim\Gin@nat@width>\linewidth\linewidth\else\Gin@nat@width\fi}
\def\maxheight{\ifdim\Gin@nat@height>\textheight\textheight\else\Gin@nat@height\fi}
\makeatother
% Scale images if necessary, so that they will not overflow the page
% margins by default, and it is still possible to overwrite the defaults
% using explicit options in \includegraphics[width, height, ...]{}
\setkeys{Gin}{width=\maxwidth,height=\maxheight,keepaspectratio}
% Set default figure placement to htbp
\makeatletter
\def\fps@figure{htbp}
\makeatother
\setlength{\emergencystretch}{3em} % prevent overfull lines
\providecommand{\tightlist}{%
  \setlength{\itemsep}{0pt}\setlength{\parskip}{0pt}}
\setcounter{secnumdepth}{-\maxdimen} % remove section numbering
\ifluatex
  \usepackage{selnolig}  % disable illegal ligatures
\fi

\title{Tarea 6}
\author{Imanol}
\date{12/3/2021}

\begin{document}
\maketitle

\hypertarget{ejercicio-1}{%
\section{EJERCICIO 1}\label{ejercicio-1}}

Resolver el siguiente sistema

\[\left\{\begin{matrix}
10x_1&+&2x_2&-&x_3&+&x_4&&&+&10x_6&=&0\\
-x_1&-&3x_2&&&&&-&x_5&+&5x_6&=&5\\
&&-x_2&+&3x_3&-&x_4&&&&&=&5\\
17x_1&+&x_2&&&+&3x_4&+&5x_5&-&15x_6&=&4\\
&&-10x_2&&&-&5x_4&+&3x_5&&&=&-21\\
-3x_1&+&x_2&+&x_3&+&x_4&-&2x_5&+&2x_6&=&11\\\end{matrix}\right.\]

Primero, comprobar el tipo de sistema (compatible determinado,
compatible indeterminado o incompatible) con R, Python y Octave.

Después, en caso de haber solución, calcularla con R, Python y Octave.
Finalmente, indicar la solución final junto con el procedimiento llevado
a cabo.

\hypertarget{r}{%
\subsection{R}\label{r}}

\begin{Shaded}
\begin{Highlighting}[]
\FunctionTok{library}\NormalTok{(matlib)}

\NormalTok{A }\OtherTok{=} \FunctionTok{rbind}\NormalTok{(}\FunctionTok{c}\NormalTok{(}\DecValTok{10}\NormalTok{,}\DecValTok{2}\NormalTok{,}\SpecialCharTok{{-}}\DecValTok{1}\NormalTok{,}\DecValTok{1}\NormalTok{,}\DecValTok{0}\NormalTok{,}\DecValTok{10}\NormalTok{),}
          \FunctionTok{c}\NormalTok{(}\SpecialCharTok{{-}}\DecValTok{1}\NormalTok{,}\SpecialCharTok{{-}}\DecValTok{3}\NormalTok{,}\DecValTok{0}\NormalTok{,}\DecValTok{0}\NormalTok{,}\SpecialCharTok{{-}}\DecValTok{1}\NormalTok{,}\DecValTok{5}\NormalTok{),}
          \FunctionTok{c}\NormalTok{(}\DecValTok{0}\NormalTok{,}\SpecialCharTok{{-}}\DecValTok{1}\NormalTok{,}\DecValTok{3}\NormalTok{,}\SpecialCharTok{{-}}\DecValTok{1}\NormalTok{,}\DecValTok{0}\NormalTok{,}\DecValTok{0}\NormalTok{),}
          \FunctionTok{c}\NormalTok{(}\DecValTok{17}\NormalTok{,}\DecValTok{1}\NormalTok{,}\DecValTok{0}\NormalTok{,}\DecValTok{3}\NormalTok{,}\DecValTok{5}\NormalTok{,}\SpecialCharTok{{-}}\DecValTok{15}\NormalTok{),}
          \FunctionTok{c}\NormalTok{(}\DecValTok{0}\NormalTok{,}\SpecialCharTok{{-}}\DecValTok{10}\NormalTok{,}\DecValTok{0}\NormalTok{,}\SpecialCharTok{{-}}\DecValTok{5}\NormalTok{,}\DecValTok{3}\NormalTok{,}\DecValTok{0}\NormalTok{),}
          \FunctionTok{c}\NormalTok{(}\SpecialCharTok{{-}}\DecValTok{3}\NormalTok{,}\DecValTok{1}\NormalTok{,}\DecValTok{1}\NormalTok{,}\DecValTok{1}\NormalTok{,}\SpecialCharTok{{-}}\DecValTok{2}\NormalTok{,}\DecValTok{2}\NormalTok{))}
\NormalTok{A}
\end{Highlighting}
\end{Shaded}

\begin{verbatim}
##      [,1] [,2] [,3] [,4] [,5] [,6]
## [1,]   10    2   -1    1    0   10
## [2,]   -1   -3    0    0   -1    5
## [3,]    0   -1    3   -1    0    0
## [4,]   17    1    0    3    5  -15
## [5,]    0  -10    0   -5    3    0
## [6,]   -3    1    1    1   -2    2
\end{verbatim}

\begin{Shaded}
\begin{Highlighting}[]
\NormalTok{b }\OtherTok{=} \FunctionTok{c}\NormalTok{(}\DecValTok{0}\NormalTok{,}\DecValTok{5}\NormalTok{,}\DecValTok{5}\NormalTok{,}\DecValTok{4}\NormalTok{,}\SpecialCharTok{{-}}\DecValTok{21}\NormalTok{,}\DecValTok{11}\NormalTok{)}
\NormalTok{b}
\end{Highlighting}
\end{Shaded}

\begin{verbatim}
## [1]   0   5   5   4 -21  11
\end{verbatim}

\begin{Shaded}
\begin{Highlighting}[]
\NormalTok{AB }\OtherTok{=}\NormalTok{ AB }\OtherTok{=} \FunctionTok{cbind}\NormalTok{(A,b)}
\NormalTok{AB}
\end{Highlighting}
\end{Shaded}

\begin{verbatim}
##                            b
## [1,] 10   2 -1  1  0  10   0
## [2,] -1  -3  0  0 -1   5   5
## [3,]  0  -1  3 -1  0   0   5
## [4,] 17   1  0  3  5 -15   4
## [5,]  0 -10  0 -5  3   0 -21
## [6,] -3   1  1  1 -2   2  11
\end{verbatim}

\begin{Shaded}
\begin{Highlighting}[]
\CommentTok{\# Comprobamos el rango}
\FunctionTok{all.equal}\NormalTok{(}\FunctionTok{R}\NormalTok{(A),}\FunctionTok{R}\NormalTok{(AB)) }\CommentTok{\# True, es compatible}
\end{Highlighting}
\end{Shaded}

\begin{verbatim}
## [1] TRUE
\end{verbatim}

\begin{Shaded}
\begin{Highlighting}[]
\FunctionTok{R}\NormalTok{(A)}\SpecialCharTok{==}\DecValTok{6} \CommentTok{\# True, es determinado. Una unica solucion}
\end{Highlighting}
\end{Shaded}

\begin{verbatim}
## [1] TRUE
\end{verbatim}

\begin{Shaded}
\begin{Highlighting}[]
\CommentTok{\# Resolvemos el sistema}
\FunctionTok{Solve}\NormalTok{(A,b)}
\end{Highlighting}
\end{Shaded}

\begin{verbatim}
## x1            =   0 
##   x2          =  -1 
##     x3        =   3 
##       x4      =   5 
##         x5    =  -2 
##           x6  =   0
\end{verbatim}

\begin{Shaded}
\begin{Highlighting}[]
  \CommentTok{\# x1 = 0}
  \CommentTok{\# x2 = {-}1}
  \CommentTok{\# x3 = 3}
  \CommentTok{\# x4 = 5}
  \CommentTok{\# x5 = {-}2}
  \CommentTok{\# x6 = 0}
\end{Highlighting}
\end{Shaded}

\hypertarget{python}{%
\subsection{Python}\label{python}}

\hypertarget{matlab}{%
\subsection{Matlab}\label{matlab}}

\hypertarget{ejercicio-2}{%
\section{EJERCICIO 2}\label{ejercicio-2}}

Resolver el siguiente sistema

\[\left\{\begin{matrix}
-2x_1&+&2x_2&-&x_3&+&x_4&&&+&4x_6&&&=&5\\
-x_1&-&3x_2&&&&&-&x_5&+&5x_6&-&2x_7&=&3\\
&&-x_2&+&3x_3&-&x_4&&&&&&&=&5\\
&&x_2&&&+&3x_4&-&2x_5&+&x_6&+&4x_7&=&0\\
&&-3x_2&&&-&5x_4&+&3x_5&&&-&2x_7&=&5\\
-3x_1&+&x_2&+&x_3&+&x_4&-&2x_5&+&2x_6&+&x_7&=&0\\\end{matrix}\right.\]

Primero, comprobar el tipo de sistema (compatible determinado,
compatible indeterminado o incompatible) con R, Python y Octave.

Después, en caso de haber solución, calcularla con R, Python y Octave.
Finalmente, indicar la solución final junto con el procedimiento llevado
a cabo.

\hypertarget{r-1}{%
\subsection{R}\label{r-1}}

\begin{Shaded}
\begin{Highlighting}[]
\FunctionTok{library}\NormalTok{(matlib)}

\NormalTok{A }\OtherTok{=} \FunctionTok{rbind}\NormalTok{(}\FunctionTok{c}\NormalTok{(}\SpecialCharTok{{-}}\DecValTok{2}\NormalTok{,}\DecValTok{2}\NormalTok{,}\SpecialCharTok{{-}}\DecValTok{1}\NormalTok{,}\DecValTok{1}\NormalTok{,}\DecValTok{0}\NormalTok{,}\DecValTok{4}\NormalTok{,}\DecValTok{0}\NormalTok{),}
          \FunctionTok{c}\NormalTok{(}\SpecialCharTok{{-}}\DecValTok{1}\NormalTok{,}\SpecialCharTok{{-}}\DecValTok{3}\NormalTok{,}\DecValTok{0}\NormalTok{,}\DecValTok{0}\NormalTok{,}\SpecialCharTok{{-}}\DecValTok{1}\NormalTok{,}\DecValTok{5}\NormalTok{,}\SpecialCharTok{{-}}\DecValTok{2}\NormalTok{),}
          \FunctionTok{c}\NormalTok{(}\DecValTok{0}\NormalTok{,}\SpecialCharTok{{-}}\DecValTok{1}\NormalTok{,}\SpecialCharTok{+}\DecValTok{3}\NormalTok{,}\SpecialCharTok{{-}}\DecValTok{1}\NormalTok{,}\DecValTok{0}\NormalTok{,}\DecValTok{0}\NormalTok{,}\DecValTok{0}\NormalTok{),}
          \FunctionTok{c}\NormalTok{(}\DecValTok{0}\NormalTok{,}\DecValTok{1}\NormalTok{,}\DecValTok{0}\NormalTok{,}\DecValTok{3}\NormalTok{,}\SpecialCharTok{{-}}\DecValTok{2}\NormalTok{,}\DecValTok{1}\NormalTok{,}\DecValTok{4}\NormalTok{),}
          \FunctionTok{c}\NormalTok{(}\DecValTok{0}\NormalTok{,}\SpecialCharTok{{-}}\DecValTok{3}\NormalTok{,}\DecValTok{0}\NormalTok{,}\SpecialCharTok{{-}}\DecValTok{5}\NormalTok{,}\DecValTok{3}\NormalTok{,}\DecValTok{0}\NormalTok{,}\SpecialCharTok{{-}}\DecValTok{2}\NormalTok{),}
          \FunctionTok{c}\NormalTok{(}\SpecialCharTok{{-}}\DecValTok{3}\NormalTok{,}\DecValTok{1}\NormalTok{,}\DecValTok{1}\NormalTok{,}\DecValTok{1}\NormalTok{,}\SpecialCharTok{{-}}\DecValTok{2}\NormalTok{,}\DecValTok{2}\NormalTok{,}\DecValTok{1}\NormalTok{))}
\NormalTok{A}
\end{Highlighting}
\end{Shaded}

\begin{verbatim}
##      [,1] [,2] [,3] [,4] [,5] [,6] [,7]
## [1,]   -2    2   -1    1    0    4    0
## [2,]   -1   -3    0    0   -1    5   -2
## [3,]    0   -1    3   -1    0    0    0
## [4,]    0    1    0    3   -2    1    4
## [5,]    0   -3    0   -5    3    0   -2
## [6,]   -3    1    1    1   -2    2    1
\end{verbatim}

\begin{Shaded}
\begin{Highlighting}[]
\NormalTok{b }\OtherTok{=} \FunctionTok{c}\NormalTok{(}\DecValTok{5}\NormalTok{,}\DecValTok{3}\NormalTok{,}\DecValTok{5}\NormalTok{,}\DecValTok{0}\NormalTok{,}\DecValTok{5}\NormalTok{,}\DecValTok{0}\NormalTok{)}
\NormalTok{b}
\end{Highlighting}
\end{Shaded}

\begin{verbatim}
## [1] 5 3 5 0 5 0
\end{verbatim}

\begin{Shaded}
\begin{Highlighting}[]
\NormalTok{AB }\OtherTok{=}\NormalTok{ AB }\OtherTok{=} \FunctionTok{cbind}\NormalTok{(A,b)}
\NormalTok{AB}
\end{Highlighting}
\end{Shaded}

\begin{verbatim}
##                          b
## [1,] -2  2 -1  1  0 4  0 5
## [2,] -1 -3  0  0 -1 5 -2 3
## [3,]  0 -1  3 -1  0 0  0 5
## [4,]  0  1  0  3 -2 1  4 0
## [5,]  0 -3  0 -5  3 0 -2 5
## [6,] -3  1  1  1 -2 2  1 0
\end{verbatim}

\begin{Shaded}
\begin{Highlighting}[]
\CommentTok{\# Comprobamos el rango}
\FunctionTok{all.equal}\NormalTok{(}\FunctionTok{R}\NormalTok{(A),}\FunctionTok{R}\NormalTok{(AB)) }\CommentTok{\# True, es compatible}
\end{Highlighting}
\end{Shaded}

\begin{verbatim}
## [1] TRUE
\end{verbatim}

\begin{Shaded}
\begin{Highlighting}[]
\FunctionTok{R}\NormalTok{(A)}\SpecialCharTok{==}\DecValTok{7} \CommentTok{\# False, es indeterminado, infinitas soluciones.}
\end{Highlighting}
\end{Shaded}

\begin{verbatim}
## [1] FALSE
\end{verbatim}

\hypertarget{python-1}{%
\subsection{Python}\label{python-1}}

\hypertarget{matlab-1}{%
\subsection{Matlab}\label{matlab-1}}

\hypertarget{ejercicio-3}{%
\section{EJERCICIO 3}\label{ejercicio-3}}

Resolver el siguiente sistema

\[\left\{\begin{matrix}
10x_1&+&2x_2&-&x_3&+&x_4&&&+&10x_6&=&0\\
-x_1&-&3x_2&&&&&-&x_5&+&5x_6&=&5\\
9x_1&-&x_2&-&x_3&+&x_4&-&x_5&+&15x_6&=&0\\
17x_1&+&x_2&&&+&3x_4&+&5x_5&-&15x_6&=&4\\
&&-10x_2&&&-&5x_4&+&3x_5&&&=&-21\\
-3x_1&+&x_2&+&x_3&+&x_4&-&2x_5&+&2x_6&=&11\\\end{matrix}\right.\]

Primero, comprobar el tipo de sistema (compatible determinado,
compatible indeterminado o incompatible) con R, Python y Octave.

Después, en caso de haber solución, calcularla con R, Python y Octave.
Finalmente, indicar la solución final junto con el procedimiento llevado
a cabo.

\hypertarget{r-2}{%
\subsection{R}\label{r-2}}

\begin{Shaded}
\begin{Highlighting}[]
\FunctionTok{library}\NormalTok{(matlib)}

\NormalTok{A }\OtherTok{=} \FunctionTok{rbind}\NormalTok{(}\FunctionTok{c}\NormalTok{(}\DecValTok{10}\NormalTok{,}\DecValTok{2}\NormalTok{,}\SpecialCharTok{{-}}\DecValTok{1}\NormalTok{,}\DecValTok{1}\NormalTok{,}\DecValTok{0}\NormalTok{,}\DecValTok{10}\NormalTok{),}
          \FunctionTok{c}\NormalTok{(}\SpecialCharTok{{-}}\DecValTok{1}\NormalTok{,}\SpecialCharTok{{-}}\DecValTok{3}\NormalTok{,}\DecValTok{0}\NormalTok{,}\DecValTok{0}\NormalTok{,}\SpecialCharTok{{-}}\DecValTok{1}\NormalTok{,}\DecValTok{5}\NormalTok{),}
          \FunctionTok{c}\NormalTok{(}\DecValTok{9}\NormalTok{,}\SpecialCharTok{{-}}\DecValTok{1}\NormalTok{,}\SpecialCharTok{{-}}\DecValTok{1}\NormalTok{,}\DecValTok{1}\NormalTok{,}\SpecialCharTok{{-}}\DecValTok{1}\NormalTok{,}\DecValTok{15}\NormalTok{),}
          \FunctionTok{c}\NormalTok{(}\DecValTok{17}\NormalTok{,}\DecValTok{1}\NormalTok{,}\DecValTok{0}\NormalTok{,}\DecValTok{3}\NormalTok{,}\DecValTok{5}\NormalTok{,}\SpecialCharTok{{-}}\DecValTok{15}\NormalTok{),}
          \FunctionTok{c}\NormalTok{(}\DecValTok{0}\NormalTok{,}\SpecialCharTok{{-}}\DecValTok{10}\NormalTok{,}\DecValTok{0}\NormalTok{,}\SpecialCharTok{{-}}\DecValTok{5}\NormalTok{,}\DecValTok{3}\NormalTok{,}\DecValTok{0}\NormalTok{),}
          \FunctionTok{c}\NormalTok{(}\SpecialCharTok{{-}}\DecValTok{3}\NormalTok{,}\DecValTok{1}\NormalTok{,}\DecValTok{1}\NormalTok{,}\DecValTok{1}\NormalTok{,}\SpecialCharTok{{-}}\DecValTok{2}\NormalTok{,}\DecValTok{2}\NormalTok{))}
\NormalTok{A}
\end{Highlighting}
\end{Shaded}

\begin{verbatim}
##      [,1] [,2] [,3] [,4] [,5] [,6]
## [1,]   10    2   -1    1    0   10
## [2,]   -1   -3    0    0   -1    5
## [3,]    9   -1   -1    1   -1   15
## [4,]   17    1    0    3    5  -15
## [5,]    0  -10    0   -5    3    0
## [6,]   -3    1    1    1   -2    2
\end{verbatim}

\begin{Shaded}
\begin{Highlighting}[]
\NormalTok{b }\OtherTok{=} \FunctionTok{c}\NormalTok{(}\DecValTok{0}\NormalTok{,}\DecValTok{5}\NormalTok{,}\DecValTok{0}\NormalTok{,}\DecValTok{4}\NormalTok{,}\SpecialCharTok{{-}}\DecValTok{21}\NormalTok{,}\DecValTok{11}\NormalTok{)}
\NormalTok{b}
\end{Highlighting}
\end{Shaded}

\begin{verbatim}
## [1]   0   5   0   4 -21  11
\end{verbatim}

\begin{Shaded}
\begin{Highlighting}[]
\NormalTok{AB }\OtherTok{=}\NormalTok{ AB }\OtherTok{=} \FunctionTok{cbind}\NormalTok{(A,b)}
\NormalTok{AB}
\end{Highlighting}
\end{Shaded}

\begin{verbatim}
##                            b
## [1,] 10   2 -1  1  0  10   0
## [2,] -1  -3  0  0 -1   5   5
## [3,]  9  -1 -1  1 -1  15   0
## [4,] 17   1  0  3  5 -15   4
## [5,]  0 -10  0 -5  3   0 -21
## [6,] -3   1  1  1 -2   2  11
\end{verbatim}

\begin{Shaded}
\begin{Highlighting}[]
\CommentTok{\# Comprobamos el rango}
\FunctionTok{all.equal}\NormalTok{(}\FunctionTok{R}\NormalTok{(A),}\FunctionTok{R}\NormalTok{(AB)) }\CommentTok{\# Hay diferencia, es Incompatible. No hay solucion.}
\end{Highlighting}
\end{Shaded}

\begin{verbatim}
## [1] "Mean relative difference: 0.2"
\end{verbatim}

\hypertarget{python-2}{%
\subsection{Python}\label{python-2}}

\hypertarget{matlab-2}{%
\subsection{Matlab}\label{matlab-2}}

\hypertarget{ejercicio-4}{%
\section{EJERCICIO 4}\label{ejercicio-4}}

Encuentra la matriz X, ya sea a mano o con R,Python u Octave, tal que

\hypertarget{apartado-a}{%
\subsection{Apartado (a)}\label{apartado-a}}

\[AX+3B = -5X\] donde las matrices \(A\) y \(B\) son:

\[A = \begin{pmatrix}-6 & -3\\ 0 & -3\end{pmatrix},\quad B = \begin{pmatrix}-1 & 0\\ 4 & -2\end{pmatrix}\]

\hypertarget{r-3}{%
\subsubsection{R}\label{r-3}}

\begin{Shaded}
\begin{Highlighting}[]
\CommentTok{\# Resuelto (A{-}5*I)X = 3*B}

\NormalTok{A }\OtherTok{=} \FunctionTok{rbind}\NormalTok{(}\FunctionTok{c}\NormalTok{(}\SpecialCharTok{{-}}\DecValTok{6}\NormalTok{,}\SpecialCharTok{{-}}\DecValTok{3}\NormalTok{),}\FunctionTok{c}\NormalTok{(}\DecValTok{0}\NormalTok{,}\SpecialCharTok{{-}}\DecValTok{3}\NormalTok{))}
\NormalTok{A}
\end{Highlighting}
\end{Shaded}

\begin{verbatim}
##      [,1] [,2]
## [1,]   -6   -3
## [2,]    0   -3
\end{verbatim}

\begin{Shaded}
\begin{Highlighting}[]
\NormalTok{B }\OtherTok{=} \FunctionTok{rbind}\NormalTok{(}\FunctionTok{c}\NormalTok{(}\SpecialCharTok{{-}}\DecValTok{1}\NormalTok{,}\DecValTok{0}\NormalTok{),}\FunctionTok{c}\NormalTok{(}\DecValTok{4}\NormalTok{,}\SpecialCharTok{{-}}\DecValTok{2}\NormalTok{))}
\NormalTok{B}
\end{Highlighting}
\end{Shaded}

\begin{verbatim}
##      [,1] [,2]
## [1,]   -1    0
## [2,]    4   -2
\end{verbatim}

\hypertarget{python-3}{%
\subsubsection{Python}\label{python-3}}

\hypertarget{matlab-3}{%
\subsubsection{Matlab}\label{matlab-3}}

\hypertarget{apartado-b}{%
\subsection{Apartado (b)}\label{apartado-b}}

\[ACX+3B = 10I-X\] donde las matrices \(A\), \(B\), \(C\) e \(I\) son:

\[A = \begin{pmatrix}1 & 2\\ 0 & 2\end{pmatrix},\quad B=\begin{pmatrix}2 & -1\\ -1 & 5\end{pmatrix},\quad C = \begin{pmatrix}2 & 6\\ -1 & -0.5\end{pmatrix},\quad I=\begin{pmatrix}1 & 0 \\0 & 1\end{pmatrix}\]

\hypertarget{r-4}{%
\subsubsection{R}\label{r-4}}

\hypertarget{python-4}{%
\subsubsection{Python}\label{python-4}}

\hypertarget{matlab-4}{%
\subsubsection{Matlab}\label{matlab-4}}

\end{document}
